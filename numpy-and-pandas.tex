\documentclass[11pt]{article}

    \usepackage[breakable]{tcolorbox}
    \usepackage{parskip} % Stop auto-indenting (to mimic markdown behaviour)
    

    % Basic figure setup, for now with no caption control since it's done
    % automatically by Pandoc (which extracts ![](path) syntax from Markdown).
    \usepackage{graphicx}
    % Maintain compatibility with old templates. Remove in nbconvert 6.0
    \let\Oldincludegraphics\includegraphics
    % Ensure that by default, figures have no caption (until we provide a
    % proper Figure object with a Caption API and a way to capture that
    % in the conversion process - todo).
    \usepackage{caption}
    \DeclareCaptionFormat{nocaption}{}
    \captionsetup{format=nocaption,aboveskip=0pt,belowskip=0pt}

    \usepackage{float}
    \floatplacement{figure}{H} % forces figures to be placed at the correct location
    \usepackage{xcolor} % Allow colors to be defined
    \usepackage{enumerate} % Needed for markdown enumerations to work
    \usepackage{geometry} % Used to adjust the document margins
    \usepackage{amsmath} % Equations
    \usepackage{amssymb} % Equations
    \usepackage{textcomp} % defines textquotesingle
    % Hack from http://tex.stackexchange.com/a/47451/13684:
    \AtBeginDocument{%
        \def\PYZsq{\textquotesingle}% Upright quotes in Pygmentized code
    }
    \usepackage{upquote} % Upright quotes for verbatim code
    \usepackage{eurosym} % defines \euro

    \usepackage{iftex}
    \ifPDFTeX
        \usepackage[T1]{fontenc}
        \IfFileExists{alphabeta.sty}{
              \usepackage{alphabeta}
          }{
              \usepackage[mathletters]{ucs}
              \usepackage[utf8x]{inputenc}
          }
    \else
        \usepackage{fontspec}
        \usepackage{unicode-math}
    \fi

    \usepackage{fancyvrb} % verbatim replacement that allows latex
    \usepackage{grffile} % extends the file name processing of package graphics
                         % to support a larger range
    \makeatletter % fix for old versions of grffile with XeLaTeX
    \@ifpackagelater{grffile}{2019/11/01}
    {
      % Do nothing on new versions
    }
    {
      \def\Gread@@xetex#1{%
        \IfFileExists{"\Gin@base".bb}%
        {\Gread@eps{\Gin@base.bb}}%
        {\Gread@@xetex@aux#1}%
      }
    }
    \makeatother
    \usepackage[Export]{adjustbox} % Used to constrain images to a maximum size
    \adjustboxset{max size={0.9\linewidth}{0.9\paperheight}}

    % The hyperref package gives us a pdf with properly built
    % internal navigation ('pdf bookmarks' for the table of contents,
    % internal cross-reference links, web links for URLs, etc.)
    \usepackage{hyperref}
    % The default LaTeX title has an obnoxious amount of whitespace. By default,
    % titling removes some of it. It also provides customization options.
    \usepackage{titling}
    \usepackage{longtable} % longtable support required by pandoc >1.10
    \usepackage{booktabs}  % table support for pandoc > 1.12.2
    \usepackage{array}     % table support for pandoc >= 2.11.3
    \usepackage{calc}      % table minipage width calculation for pandoc >= 2.11.1
    \usepackage[inline]{enumitem} % IRkernel/repr support (it uses the enumerate* environment)
    \usepackage[normalem]{ulem} % ulem is needed to support strikethroughs (\sout)
                                % normalem makes italics be italics, not underlines
    \usepackage{mathrsfs}
    

    
    % Colors for the hyperref package
    \definecolor{urlcolor}{rgb}{0,.145,.698}
    \definecolor{linkcolor}{rgb}{.71,0.21,0.01}
    \definecolor{citecolor}{rgb}{.12,.54,.11}

    % ANSI colors
    \definecolor{ansi-black}{HTML}{3E424D}
    \definecolor{ansi-black-intense}{HTML}{282C36}
    \definecolor{ansi-red}{HTML}{E75C58}
    \definecolor{ansi-red-intense}{HTML}{B22B31}
    \definecolor{ansi-green}{HTML}{00A250}
    \definecolor{ansi-green-intense}{HTML}{007427}
    \definecolor{ansi-yellow}{HTML}{DDB62B}
    \definecolor{ansi-yellow-intense}{HTML}{B27D12}
    \definecolor{ansi-blue}{HTML}{208FFB}
    \definecolor{ansi-blue-intense}{HTML}{0065CA}
    \definecolor{ansi-magenta}{HTML}{D160C4}
    \definecolor{ansi-magenta-intense}{HTML}{A03196}
    \definecolor{ansi-cyan}{HTML}{60C6C8}
    \definecolor{ansi-cyan-intense}{HTML}{258F8F}
    \definecolor{ansi-white}{HTML}{C5C1B4}
    \definecolor{ansi-white-intense}{HTML}{A1A6B2}
    \definecolor{ansi-default-inverse-fg}{HTML}{FFFFFF}
    \definecolor{ansi-default-inverse-bg}{HTML}{000000}

    % common color for the border for error outputs.
    \definecolor{outerrorbackground}{HTML}{FFDFDF}

    % commands and environments needed by pandoc snippets
    % extracted from the output of `pandoc -s`
    \providecommand{\tightlist}{%
      \setlength{\itemsep}{0pt}\setlength{\parskip}{0pt}}
    \DefineVerbatimEnvironment{Highlighting}{Verbatim}{commandchars=\\\{\}}
    % Add ',fontsize=\small' for more characters per line
    \newenvironment{Shaded}{}{}
    \newcommand{\KeywordTok}[1]{\textcolor[rgb]{0.00,0.44,0.13}{\textbf{{#1}}}}
    \newcommand{\DataTypeTok}[1]{\textcolor[rgb]{0.56,0.13,0.00}{{#1}}}
    \newcommand{\DecValTok}[1]{\textcolor[rgb]{0.25,0.63,0.44}{{#1}}}
    \newcommand{\BaseNTok}[1]{\textcolor[rgb]{0.25,0.63,0.44}{{#1}}}
    \newcommand{\FloatTok}[1]{\textcolor[rgb]{0.25,0.63,0.44}{{#1}}}
    \newcommand{\CharTok}[1]{\textcolor[rgb]{0.25,0.44,0.63}{{#1}}}
    \newcommand{\StringTok}[1]{\textcolor[rgb]{0.25,0.44,0.63}{{#1}}}
    \newcommand{\CommentTok}[1]{\textcolor[rgb]{0.38,0.63,0.69}{\textit{{#1}}}}
    \newcommand{\OtherTok}[1]{\textcolor[rgb]{0.00,0.44,0.13}{{#1}}}
    \newcommand{\AlertTok}[1]{\textcolor[rgb]{1.00,0.00,0.00}{\textbf{{#1}}}}
    \newcommand{\FunctionTok}[1]{\textcolor[rgb]{0.02,0.16,0.49}{{#1}}}
    \newcommand{\RegionMarkerTok}[1]{{#1}}
    \newcommand{\ErrorTok}[1]{\textcolor[rgb]{1.00,0.00,0.00}{\textbf{{#1}}}}
    \newcommand{\NormalTok}[1]{{#1}}

    % Additional commands for more recent versions of Pandoc
    \newcommand{\ConstantTok}[1]{\textcolor[rgb]{0.53,0.00,0.00}{{#1}}}
    \newcommand{\SpecialCharTok}[1]{\textcolor[rgb]{0.25,0.44,0.63}{{#1}}}
    \newcommand{\VerbatimStringTok}[1]{\textcolor[rgb]{0.25,0.44,0.63}{{#1}}}
    \newcommand{\SpecialStringTok}[1]{\textcolor[rgb]{0.73,0.40,0.53}{{#1}}}
    \newcommand{\ImportTok}[1]{{#1}}
    \newcommand{\DocumentationTok}[1]{\textcolor[rgb]{0.73,0.13,0.13}{\textit{{#1}}}}
    \newcommand{\AnnotationTok}[1]{\textcolor[rgb]{0.38,0.63,0.69}{\textbf{\textit{{#1}}}}}
    \newcommand{\CommentVarTok}[1]{\textcolor[rgb]{0.38,0.63,0.69}{\textbf{\textit{{#1}}}}}
    \newcommand{\VariableTok}[1]{\textcolor[rgb]{0.10,0.09,0.49}{{#1}}}
    \newcommand{\ControlFlowTok}[1]{\textcolor[rgb]{0.00,0.44,0.13}{\textbf{{#1}}}}
    \newcommand{\OperatorTok}[1]{\textcolor[rgb]{0.40,0.40,0.40}{{#1}}}
    \newcommand{\BuiltInTok}[1]{{#1}}
    \newcommand{\ExtensionTok}[1]{{#1}}
    \newcommand{\PreprocessorTok}[1]{\textcolor[rgb]{0.74,0.48,0.00}{{#1}}}
    \newcommand{\AttributeTok}[1]{\textcolor[rgb]{0.49,0.56,0.16}{{#1}}}
    \newcommand{\InformationTok}[1]{\textcolor[rgb]{0.38,0.63,0.69}{\textbf{\textit{{#1}}}}}
    \newcommand{\WarningTok}[1]{\textcolor[rgb]{0.38,0.63,0.69}{\textbf{\textit{{#1}}}}}


    % Define a nice break command that doesn't care if a line doesn't already
    % exist.
    \def\br{\hspace*{\fill} \\* }
    % Math Jax compatibility definitions
    \def\gt{>}
    \def\lt{<}
    \let\Oldtex\TeX
    \let\Oldlatex\LaTeX
    \renewcommand{\TeX}{\textrm{\Oldtex}}
    \renewcommand{\LaTeX}{\textrm{\Oldlatex}}
    % Document parameters
    % Document title
    \title{Notebook}
    
    
    
    
    
    
    
% Pygments definitions
\makeatletter
\def\PY@reset{\let\PY@it=\relax \let\PY@bf=\relax%
    \let\PY@ul=\relax \let\PY@tc=\relax%
    \let\PY@bc=\relax \let\PY@ff=\relax}
\def\PY@tok#1{\csname PY@tok@#1\endcsname}
\def\PY@toks#1+{\ifx\relax#1\empty\else%
    \PY@tok{#1}\expandafter\PY@toks\fi}
\def\PY@do#1{\PY@bc{\PY@tc{\PY@ul{%
    \PY@it{\PY@bf{\PY@ff{#1}}}}}}}
\def\PY#1#2{\PY@reset\PY@toks#1+\relax+\PY@do{#2}}

\@namedef{PY@tok@w}{\def\PY@tc##1{\textcolor[rgb]{0.73,0.73,0.73}{##1}}}
\@namedef{PY@tok@c}{\let\PY@it=\textit\def\PY@tc##1{\textcolor[rgb]{0.24,0.48,0.48}{##1}}}
\@namedef{PY@tok@cp}{\def\PY@tc##1{\textcolor[rgb]{0.61,0.40,0.00}{##1}}}
\@namedef{PY@tok@k}{\let\PY@bf=\textbf\def\PY@tc##1{\textcolor[rgb]{0.00,0.50,0.00}{##1}}}
\@namedef{PY@tok@kp}{\def\PY@tc##1{\textcolor[rgb]{0.00,0.50,0.00}{##1}}}
\@namedef{PY@tok@kt}{\def\PY@tc##1{\textcolor[rgb]{0.69,0.00,0.25}{##1}}}
\@namedef{PY@tok@o}{\def\PY@tc##1{\textcolor[rgb]{0.40,0.40,0.40}{##1}}}
\@namedef{PY@tok@ow}{\let\PY@bf=\textbf\def\PY@tc##1{\textcolor[rgb]{0.67,0.13,1.00}{##1}}}
\@namedef{PY@tok@nb}{\def\PY@tc##1{\textcolor[rgb]{0.00,0.50,0.00}{##1}}}
\@namedef{PY@tok@nf}{\def\PY@tc##1{\textcolor[rgb]{0.00,0.00,1.00}{##1}}}
\@namedef{PY@tok@nc}{\let\PY@bf=\textbf\def\PY@tc##1{\textcolor[rgb]{0.00,0.00,1.00}{##1}}}
\@namedef{PY@tok@nn}{\let\PY@bf=\textbf\def\PY@tc##1{\textcolor[rgb]{0.00,0.00,1.00}{##1}}}
\@namedef{PY@tok@ne}{\let\PY@bf=\textbf\def\PY@tc##1{\textcolor[rgb]{0.80,0.25,0.22}{##1}}}
\@namedef{PY@tok@nv}{\def\PY@tc##1{\textcolor[rgb]{0.10,0.09,0.49}{##1}}}
\@namedef{PY@tok@no}{\def\PY@tc##1{\textcolor[rgb]{0.53,0.00,0.00}{##1}}}
\@namedef{PY@tok@nl}{\def\PY@tc##1{\textcolor[rgb]{0.46,0.46,0.00}{##1}}}
\@namedef{PY@tok@ni}{\let\PY@bf=\textbf\def\PY@tc##1{\textcolor[rgb]{0.44,0.44,0.44}{##1}}}
\@namedef{PY@tok@na}{\def\PY@tc##1{\textcolor[rgb]{0.41,0.47,0.13}{##1}}}
\@namedef{PY@tok@nt}{\let\PY@bf=\textbf\def\PY@tc##1{\textcolor[rgb]{0.00,0.50,0.00}{##1}}}
\@namedef{PY@tok@nd}{\def\PY@tc##1{\textcolor[rgb]{0.67,0.13,1.00}{##1}}}
\@namedef{PY@tok@s}{\def\PY@tc##1{\textcolor[rgb]{0.73,0.13,0.13}{##1}}}
\@namedef{PY@tok@sd}{\let\PY@it=\textit\def\PY@tc##1{\textcolor[rgb]{0.73,0.13,0.13}{##1}}}
\@namedef{PY@tok@si}{\let\PY@bf=\textbf\def\PY@tc##1{\textcolor[rgb]{0.64,0.35,0.47}{##1}}}
\@namedef{PY@tok@se}{\let\PY@bf=\textbf\def\PY@tc##1{\textcolor[rgb]{0.67,0.36,0.12}{##1}}}
\@namedef{PY@tok@sr}{\def\PY@tc##1{\textcolor[rgb]{0.64,0.35,0.47}{##1}}}
\@namedef{PY@tok@ss}{\def\PY@tc##1{\textcolor[rgb]{0.10,0.09,0.49}{##1}}}
\@namedef{PY@tok@sx}{\def\PY@tc##1{\textcolor[rgb]{0.00,0.50,0.00}{##1}}}
\@namedef{PY@tok@m}{\def\PY@tc##1{\textcolor[rgb]{0.40,0.40,0.40}{##1}}}
\@namedef{PY@tok@gh}{\let\PY@bf=\textbf\def\PY@tc##1{\textcolor[rgb]{0.00,0.00,0.50}{##1}}}
\@namedef{PY@tok@gu}{\let\PY@bf=\textbf\def\PY@tc##1{\textcolor[rgb]{0.50,0.00,0.50}{##1}}}
\@namedef{PY@tok@gd}{\def\PY@tc##1{\textcolor[rgb]{0.63,0.00,0.00}{##1}}}
\@namedef{PY@tok@gi}{\def\PY@tc##1{\textcolor[rgb]{0.00,0.52,0.00}{##1}}}
\@namedef{PY@tok@gr}{\def\PY@tc##1{\textcolor[rgb]{0.89,0.00,0.00}{##1}}}
\@namedef{PY@tok@ge}{\let\PY@it=\textit}
\@namedef{PY@tok@gs}{\let\PY@bf=\textbf}
\@namedef{PY@tok@gp}{\let\PY@bf=\textbf\def\PY@tc##1{\textcolor[rgb]{0.00,0.00,0.50}{##1}}}
\@namedef{PY@tok@go}{\def\PY@tc##1{\textcolor[rgb]{0.44,0.44,0.44}{##1}}}
\@namedef{PY@tok@gt}{\def\PY@tc##1{\textcolor[rgb]{0.00,0.27,0.87}{##1}}}
\@namedef{PY@tok@err}{\def\PY@bc##1{{\setlength{\fboxsep}{\string -\fboxrule}\fcolorbox[rgb]{1.00,0.00,0.00}{1,1,1}{\strut ##1}}}}
\@namedef{PY@tok@kc}{\let\PY@bf=\textbf\def\PY@tc##1{\textcolor[rgb]{0.00,0.50,0.00}{##1}}}
\@namedef{PY@tok@kd}{\let\PY@bf=\textbf\def\PY@tc##1{\textcolor[rgb]{0.00,0.50,0.00}{##1}}}
\@namedef{PY@tok@kn}{\let\PY@bf=\textbf\def\PY@tc##1{\textcolor[rgb]{0.00,0.50,0.00}{##1}}}
\@namedef{PY@tok@kr}{\let\PY@bf=\textbf\def\PY@tc##1{\textcolor[rgb]{0.00,0.50,0.00}{##1}}}
\@namedef{PY@tok@bp}{\def\PY@tc##1{\textcolor[rgb]{0.00,0.50,0.00}{##1}}}
\@namedef{PY@tok@fm}{\def\PY@tc##1{\textcolor[rgb]{0.00,0.00,1.00}{##1}}}
\@namedef{PY@tok@vc}{\def\PY@tc##1{\textcolor[rgb]{0.10,0.09,0.49}{##1}}}
\@namedef{PY@tok@vg}{\def\PY@tc##1{\textcolor[rgb]{0.10,0.09,0.49}{##1}}}
\@namedef{PY@tok@vi}{\def\PY@tc##1{\textcolor[rgb]{0.10,0.09,0.49}{##1}}}
\@namedef{PY@tok@vm}{\def\PY@tc##1{\textcolor[rgb]{0.10,0.09,0.49}{##1}}}
\@namedef{PY@tok@sa}{\def\PY@tc##1{\textcolor[rgb]{0.73,0.13,0.13}{##1}}}
\@namedef{PY@tok@sb}{\def\PY@tc##1{\textcolor[rgb]{0.73,0.13,0.13}{##1}}}
\@namedef{PY@tok@sc}{\def\PY@tc##1{\textcolor[rgb]{0.73,0.13,0.13}{##1}}}
\@namedef{PY@tok@dl}{\def\PY@tc##1{\textcolor[rgb]{0.73,0.13,0.13}{##1}}}
\@namedef{PY@tok@s2}{\def\PY@tc##1{\textcolor[rgb]{0.73,0.13,0.13}{##1}}}
\@namedef{PY@tok@sh}{\def\PY@tc##1{\textcolor[rgb]{0.73,0.13,0.13}{##1}}}
\@namedef{PY@tok@s1}{\def\PY@tc##1{\textcolor[rgb]{0.73,0.13,0.13}{##1}}}
\@namedef{PY@tok@mb}{\def\PY@tc##1{\textcolor[rgb]{0.40,0.40,0.40}{##1}}}
\@namedef{PY@tok@mf}{\def\PY@tc##1{\textcolor[rgb]{0.40,0.40,0.40}{##1}}}
\@namedef{PY@tok@mh}{\def\PY@tc##1{\textcolor[rgb]{0.40,0.40,0.40}{##1}}}
\@namedef{PY@tok@mi}{\def\PY@tc##1{\textcolor[rgb]{0.40,0.40,0.40}{##1}}}
\@namedef{PY@tok@il}{\def\PY@tc##1{\textcolor[rgb]{0.40,0.40,0.40}{##1}}}
\@namedef{PY@tok@mo}{\def\PY@tc##1{\textcolor[rgb]{0.40,0.40,0.40}{##1}}}
\@namedef{PY@tok@ch}{\let\PY@it=\textit\def\PY@tc##1{\textcolor[rgb]{0.24,0.48,0.48}{##1}}}
\@namedef{PY@tok@cm}{\let\PY@it=\textit\def\PY@tc##1{\textcolor[rgb]{0.24,0.48,0.48}{##1}}}
\@namedef{PY@tok@cpf}{\let\PY@it=\textit\def\PY@tc##1{\textcolor[rgb]{0.24,0.48,0.48}{##1}}}
\@namedef{PY@tok@c1}{\let\PY@it=\textit\def\PY@tc##1{\textcolor[rgb]{0.24,0.48,0.48}{##1}}}
\@namedef{PY@tok@cs}{\let\PY@it=\textit\def\PY@tc##1{\textcolor[rgb]{0.24,0.48,0.48}{##1}}}

\def\PYZbs{\char`\\}
\def\PYZus{\char`\_}
\def\PYZob{\char`\{}
\def\PYZcb{\char`\}}
\def\PYZca{\char`\^}
\def\PYZam{\char`\&}
\def\PYZlt{\char`\<}
\def\PYZgt{\char`\>}
\def\PYZsh{\char`\#}
\def\PYZpc{\char`\%}
\def\PYZdl{\char`\$}
\def\PYZhy{\char`\-}
\def\PYZsq{\char`\'}
\def\PYZdq{\char`\"}
\def\PYZti{\char`\~}
% for compatibility with earlier versions
\def\PYZat{@}
\def\PYZlb{[}
\def\PYZrb{]}
\makeatother


    % For linebreaks inside Verbatim environment from package fancyvrb.
    \makeatletter
        \newbox\Wrappedcontinuationbox
        \newbox\Wrappedvisiblespacebox
        \newcommand*\Wrappedvisiblespace {\textcolor{red}{\textvisiblespace}}
        \newcommand*\Wrappedcontinuationsymbol {\textcolor{red}{\llap{\tiny$\m@th\hookrightarrow$}}}
        \newcommand*\Wrappedcontinuationindent {3ex }
        \newcommand*\Wrappedafterbreak {\kern\Wrappedcontinuationindent\copy\Wrappedcontinuationbox}
        % Take advantage of the already applied Pygments mark-up to insert
        % potential linebreaks for TeX processing.
        %        {, <, #, %, $, ' and ": go to next line.
        %        _, }, ^, &, >, - and ~: stay at end of broken line.
        % Use of \textquotesingle for straight quote.
        \newcommand*\Wrappedbreaksatspecials {%
            \def\PYGZus{\discretionary{\char`\_}{\Wrappedafterbreak}{\char`\_}}%
            \def\PYGZob{\discretionary{}{\Wrappedafterbreak\char`\{}{\char`\{}}%
            \def\PYGZcb{\discretionary{\char`\}}{\Wrappedafterbreak}{\char`\}}}%
            \def\PYGZca{\discretionary{\char`\^}{\Wrappedafterbreak}{\char`\^}}%
            \def\PYGZam{\discretionary{\char`\&}{\Wrappedafterbreak}{\char`\&}}%
            \def\PYGZlt{\discretionary{}{\Wrappedafterbreak\char`\<}{\char`\<}}%
            \def\PYGZgt{\discretionary{\char`\>}{\Wrappedafterbreak}{\char`\>}}%
            \def\PYGZsh{\discretionary{}{\Wrappedafterbreak\char`\#}{\char`\#}}%
            \def\PYGZpc{\discretionary{}{\Wrappedafterbreak\char`\%}{\char`\%}}%
            \def\PYGZdl{\discretionary{}{\Wrappedafterbreak\char`\$}{\char`\$}}%
            \def\PYGZhy{\discretionary{\char`\-}{\Wrappedafterbreak}{\char`\-}}%
            \def\PYGZsq{\discretionary{}{\Wrappedafterbreak\textquotesingle}{\textquotesingle}}%
            \def\PYGZdq{\discretionary{}{\Wrappedafterbreak\char`\"}{\char`\"}}%
            \def\PYGZti{\discretionary{\char`\~}{\Wrappedafterbreak}{\char`\~}}%
        }
        % Some characters . , ; ? ! / are not pygmentized.
        % This macro makes them "active" and they will insert potential linebreaks
        \newcommand*\Wrappedbreaksatpunct {%
            \lccode`\~`\.\lowercase{\def~}{\discretionary{\hbox{\char`\.}}{\Wrappedafterbreak}{\hbox{\char`\.}}}%
            \lccode`\~`\,\lowercase{\def~}{\discretionary{\hbox{\char`\,}}{\Wrappedafterbreak}{\hbox{\char`\,}}}%
            \lccode`\~`\;\lowercase{\def~}{\discretionary{\hbox{\char`\;}}{\Wrappedafterbreak}{\hbox{\char`\;}}}%
            \lccode`\~`\:\lowercase{\def~}{\discretionary{\hbox{\char`\:}}{\Wrappedafterbreak}{\hbox{\char`\:}}}%
            \lccode`\~`\?\lowercase{\def~}{\discretionary{\hbox{\char`\?}}{\Wrappedafterbreak}{\hbox{\char`\?}}}%
            \lccode`\~`\!\lowercase{\def~}{\discretionary{\hbox{\char`\!}}{\Wrappedafterbreak}{\hbox{\char`\!}}}%
            \lccode`\~`\/\lowercase{\def~}{\discretionary{\hbox{\char`\/}}{\Wrappedafterbreak}{\hbox{\char`\/}}}%
            \catcode`\.\active
            \catcode`\,\active
            \catcode`\;\active
            \catcode`\:\active
            \catcode`\?\active
            \catcode`\!\active
            \catcode`\/\active
            \lccode`\~`\~
        }
    \makeatother

    \let\OriginalVerbatim=\Verbatim
    \makeatletter
    \renewcommand{\Verbatim}[1][1]{%
        %\parskip\z@skip
        \sbox\Wrappedcontinuationbox {\Wrappedcontinuationsymbol}%
        \sbox\Wrappedvisiblespacebox {\FV@SetupFont\Wrappedvisiblespace}%
        \def\FancyVerbFormatLine ##1{\hsize\linewidth
            \vtop{\raggedright\hyphenpenalty\z@\exhyphenpenalty\z@
                \doublehyphendemerits\z@\finalhyphendemerits\z@
                \strut ##1\strut}%
        }%
        % If the linebreak is at a space, the latter will be displayed as visible
        % space at end of first line, and a continuation symbol starts next line.
        % Stretch/shrink are however usually zero for typewriter font.
        \def\FV@Space {%
            \nobreak\hskip\z@ plus\fontdimen3\font minus\fontdimen4\font
            \discretionary{\copy\Wrappedvisiblespacebox}{\Wrappedafterbreak}
            {\kern\fontdimen2\font}%
        }%

        % Allow breaks at special characters using \PYG... macros.
        \Wrappedbreaksatspecials
        % Breaks at punctuation characters . , ; ? ! and / need catcode=\active
        \OriginalVerbatim[#1,codes*=\Wrappedbreaksatpunct]%
    }
    \makeatother

    % Exact colors from NB
    \definecolor{incolor}{HTML}{303F9F}
    \definecolor{outcolor}{HTML}{D84315}
    \definecolor{cellborder}{HTML}{CFCFCF}
    \definecolor{cellbackground}{HTML}{F7F7F7}

    % prompt
    \makeatletter
    \newcommand{\boxspacing}{\kern\kvtcb@left@rule\kern\kvtcb@boxsep}
    \makeatother
    \newcommand{\prompt}[4]{
        {\ttfamily\llap{{\color{#2}[#3]:\hspace{3pt}#4}}\vspace{-\baselineskip}}
    }
    

    
    % Prevent overflowing lines due to hard-to-break entities
    \sloppy
    % Setup hyperref package
    \hypersetup{
      breaklinks=true,  % so long urls are correctly broken across lines
      colorlinks=true,
      urlcolor=urlcolor,
      linkcolor=linkcolor,
      citecolor=citecolor,
      }
    % Slightly bigger margins than the latex defaults
    
    \geometry{verbose,tmargin=1in,bmargin=1in,lmargin=1in,rmargin=1in}
    
    

\begin{document}
    
    \maketitle
    
    

    
    \hypertarget{exploring-data-with-python}{%
\section{Exploring Data with Python}\label{exploring-data-with-python}}

A significant part of a a data scientist's role is to explore, analyze,
and visualize data. There are many tools and programming languages that
they can use to do this. One of the most popular approaches is to use
Jupyter notebooks (like this one) and Python.

Python is a flexible programming language that's used in a wide range of
scenarios, from web applications to device programming. It's extremely
popular in the data-science and machine-learning communities because of
the many packages it supports for data analysis and visualization.

In this notebook, we'll explore some of these packages and apply basic
techniques to analyze data. This isn't intended to be a comprehensive
Python programming exercise or even a deep dive into data analysis.
Rather, it's intended as a crash course in some of the common ways in
which data scientists can use Python to work with data.

\begin{quote}
\textbf{Note}: If you've never used the Jupyter Notebooks environment
before, there are a few things of which you should be aware:

\begin{itemize}
\tightlist
\item
  Notebooks are made up of \emph{cells}. Some cells (like this one)
  contain \emph{markdown} text, while others (like the one following
  this one) contain code.
\item
  You can run each code cell by using the \textbf{► Run} button. The
  \textbf{► Run} button shows up when you hover over the cell.
\item
  The output from each code cell is displayed immediately below the
  cell.
\item
  Even though you can run the code cells individually, some variables
  the code uses are global to the notebook. That means that you should
  run all of the code cells \textbf{in order}. There might be
  dependencies between code cells, so if you skip a cell, subsequent
  cells might not run correctly.
\end{itemize}
\end{quote}

\hypertarget{exploring-data-arrays-with-numpy}{%
\subsection{Exploring data arrays with
NumPy}\label{exploring-data-arrays-with-numpy}}

Let's start by looking at some simple data.

Suppose a college professor takes a sample of student grades from a
class to analyze.

Run the code in the following cell by selecting the \textbf{► Run}
button to see the data.

    \begin{tcolorbox}[breakable, size=fbox, boxrule=1pt, pad at break*=1mm,colback=cellbackground, colframe=cellborder]
\prompt{In}{incolor}{27}{\boxspacing}
\begin{Verbatim}[commandchars=\\\{\}]
\PY{n}{data} \PY{o}{=} \PY{p}{[}\PY{l+m+mi}{50}\PY{p}{,}\PY{l+m+mi}{50}\PY{p}{,}\PY{l+m+mi}{47}\PY{p}{,}\PY{l+m+mi}{97}\PY{p}{,}\PY{l+m+mi}{49}\PY{p}{,}\PY{l+m+mi}{3}\PY{p}{,}\PY{l+m+mi}{53}\PY{p}{,}\PY{l+m+mi}{42}\PY{p}{,}\PY{l+m+mi}{26}\PY{p}{,}\PY{l+m+mi}{74}\PY{p}{,}\PY{l+m+mi}{82}\PY{p}{,}\PY{l+m+mi}{62}\PY{p}{,}\PY{l+m+mi}{37}\PY{p}{,}\PY{l+m+mi}{15}\PY{p}{,}\PY{l+m+mi}{70}\PY{p}{,}\PY{l+m+mi}{27}\PY{p}{,}\PY{l+m+mi}{36}\PY{p}{,}\PY{l+m+mi}{35}\PY{p}{,}\PY{l+m+mi}{48}\PY{p}{,}\PY{l+m+mi}{52}\PY{p}{,}\PY{l+m+mi}{63}\PY{p}{,}\PY{l+m+mi}{64}\PY{p}{]}
\PY{n+nb}{print}\PY{p}{(}\PY{n}{data}\PY{p}{)}
\end{Verbatim}
\end{tcolorbox}

    \begin{Verbatim}[commandchars=\\\{\}]
[50, 50, 47, 97, 49, 3, 53, 42, 26, 74, 82, 62, 37, 15, 70, 27, 36, 35, 48, 52,
63, 64]
    \end{Verbatim}

    The data has been loaded into a Python \textbf{list} structure, which is
a good data type for general data manipulation, but it's not optimized
for numeric analysis. For that, we're going to use the \textbf{NumPy}
package, which includes specific data types and functions for working
with \emph{Num}bers in \emph{Py}thon.

Run the following cell to load the data into a NumPy \textbf{array}.

    \begin{tcolorbox}[breakable, size=fbox, boxrule=1pt, pad at break*=1mm,colback=cellbackground, colframe=cellborder]
\prompt{In}{incolor}{28}{\boxspacing}
\begin{Verbatim}[commandchars=\\\{\}]
\PY{k+kn}{import} \PY{n+nn}{numpy} \PY{k}{as} \PY{n+nn}{np}

\PY{n}{grades} \PY{o}{=} \PY{n}{np}\PY{o}{.}\PY{n}{array}\PY{p}{(}\PY{n}{data}\PY{p}{)}
\PY{n+nb}{print}\PY{p}{(}\PY{n}{grades}\PY{p}{)}
\end{Verbatim}
\end{tcolorbox}

    \begin{Verbatim}[commandchars=\\\{\}]
[50 50 47 97 49  3 53 42 26 74 82 62 37 15 70 27 36 35 48 52 63 64]
    \end{Verbatim}

    Just in case you're wondering about the differences between a
\textbf{list} and a NumPy \textbf{array}, let's compare how these data
types behave when we use them in an expression that multiplies them by
two.

    \begin{tcolorbox}[breakable, size=fbox, boxrule=1pt, pad at break*=1mm,colback=cellbackground, colframe=cellborder]
\prompt{In}{incolor}{29}{\boxspacing}
\begin{Verbatim}[commandchars=\\\{\}]
\PY{n+nb}{print} \PY{p}{(}\PY{n+nb}{type}\PY{p}{(}\PY{n}{data}\PY{p}{)}\PY{p}{,}\PY{l+s+s1}{\PYZsq{}}\PY{l+s+s1}{x 2:}\PY{l+s+s1}{\PYZsq{}}\PY{p}{,} \PY{n}{data} \PY{o}{*} \PY{l+m+mi}{2}\PY{p}{)}
\PY{n+nb}{print}\PY{p}{(}\PY{l+s+s1}{\PYZsq{}}\PY{l+s+s1}{\PYZhy{}\PYZhy{}\PYZhy{}}\PY{l+s+s1}{\PYZsq{}}\PY{p}{)}
\PY{n+nb}{print} \PY{p}{(}\PY{n+nb}{type}\PY{p}{(}\PY{n}{grades}\PY{p}{)}\PY{p}{,}\PY{l+s+s1}{\PYZsq{}}\PY{l+s+s1}{x 2:}\PY{l+s+s1}{\PYZsq{}}\PY{p}{,} \PY{n}{grades} \PY{o}{*} \PY{l+m+mi}{2}\PY{p}{)}
\end{Verbatim}
\end{tcolorbox}

    \begin{Verbatim}[commandchars=\\\{\}]
<class 'list'> x 2: [50, 50, 47, 97, 49, 3, 53, 42, 26, 74, 82, 62, 37, 15, 70,
27, 36, 35, 48, 52, 63, 64, 50, 50, 47, 97, 49, 3, 53, 42, 26, 74, 82, 62, 37,
15, 70, 27, 36, 35, 48, 52, 63, 64]
---
<class 'numpy.ndarray'> x 2: [100 100  94 194  98   6 106  84  52 148 164 124
74  30 140  54  72  70
  96 104 126 128]
    \end{Verbatim}

    Note that multiplying a list by two creates a new list of twice the
length with the original sequence of list elements repeated. Multiplying
a NumPy array, on the other hand, performs an element-wise calculation
in which the array behaves like a \emph{vector}, so we end up with an
array of the same size in which each element has been multiplied by two.

The key takeaway from this is that NumPy arrays are specifically
designed to support mathematical operations on numeric dat, which makes
them more useful for data analysis than a generic list.

You might have spotted that the class type for the NumPy array above is
a \textbf{numpy.ndarray}. The \textbf{nd} indicates that this is a
structure that can consist of multiple \emph{dimensions}. (It can have
\emph{n} dimensions.) Our specific instance has a single dimension of
student grades.

Run the following cell to view the \textbf{shape} of the array.

    \begin{tcolorbox}[breakable, size=fbox, boxrule=1pt, pad at break*=1mm,colback=cellbackground, colframe=cellborder]
\prompt{In}{incolor}{30}{\boxspacing}
\begin{Verbatim}[commandchars=\\\{\}]
\PY{n}{grades}\PY{o}{.}\PY{n}{shape}
\end{Verbatim}
\end{tcolorbox}

            \begin{tcolorbox}[breakable, size=fbox, boxrule=.5pt, pad at break*=1mm, opacityfill=0]
\prompt{Out}{outcolor}{30}{\boxspacing}
\begin{Verbatim}[commandchars=\\\{\}]
(22,)
\end{Verbatim}
\end{tcolorbox}
        
    The shape confirms that this array has only one dimension, which
contains 22 elements. (There are 22 grades in the original list.) You
can access the individual elements in the array by their zero-based
ordinal position. Let's get the first element (the one in position 0).

    \begin{tcolorbox}[breakable, size=fbox, boxrule=1pt, pad at break*=1mm,colback=cellbackground, colframe=cellborder]
\prompt{In}{incolor}{31}{\boxspacing}
\begin{Verbatim}[commandchars=\\\{\}]
\PY{n}{grades}\PY{p}{[}\PY{l+m+mi}{0}\PY{p}{]}
\end{Verbatim}
\end{tcolorbox}

            \begin{tcolorbox}[breakable, size=fbox, boxrule=.5pt, pad at break*=1mm, opacityfill=0]
\prompt{Out}{outcolor}{31}{\boxspacing}
\begin{Verbatim}[commandchars=\\\{\}]
50
\end{Verbatim}
\end{tcolorbox}
        
    Now that you know your way around a NumPy array, it's time to perform
some analysis of the grades data.

You can apply aggregations across the elements in the array, so let's
find the simple average grade (in other words, the \emph{mean} grade
value).

    \begin{tcolorbox}[breakable, size=fbox, boxrule=1pt, pad at break*=1mm,colback=cellbackground, colframe=cellborder]
\prompt{In}{incolor}{32}{\boxspacing}
\begin{Verbatim}[commandchars=\\\{\}]
\PY{n}{grades}\PY{o}{.}\PY{n}{mean}\PY{p}{(}\PY{p}{)}
\end{Verbatim}
\end{tcolorbox}

            \begin{tcolorbox}[breakable, size=fbox, boxrule=.5pt, pad at break*=1mm, opacityfill=0]
\prompt{Out}{outcolor}{32}{\boxspacing}
\begin{Verbatim}[commandchars=\\\{\}]
49.18181818181818
\end{Verbatim}
\end{tcolorbox}
        
    So the mean grade is just around 50, more or less in the middle of the
possible range from 0 to 100.

Let's add a second set of data for the same students. This time, we'll
record the typical number of hours per week they devoted to studying.

    \begin{tcolorbox}[breakable, size=fbox, boxrule=1pt, pad at break*=1mm,colback=cellbackground, colframe=cellborder]
\prompt{In}{incolor}{33}{\boxspacing}
\begin{Verbatim}[commandchars=\\\{\}]
\PY{c+c1}{\PYZsh{} Define an array of study hours}
\PY{n}{study\PYZus{}hours} \PY{o}{=} \PY{p}{[}\PY{l+m+mf}{10.0}\PY{p}{,}\PY{l+m+mf}{11.5}\PY{p}{,}\PY{l+m+mf}{9.0}\PY{p}{,}\PY{l+m+mf}{16.0}\PY{p}{,}\PY{l+m+mf}{9.25}\PY{p}{,}\PY{l+m+mf}{1.0}\PY{p}{,}\PY{l+m+mf}{11.5}\PY{p}{,}\PY{l+m+mf}{9.0}\PY{p}{,}\PY{l+m+mf}{8.5}\PY{p}{,}\PY{l+m+mf}{14.5}\PY{p}{,}\PY{l+m+mf}{15.5}\PY{p}{,}
               \PY{l+m+mf}{13.75}\PY{p}{,}\PY{l+m+mf}{9.0}\PY{p}{,}\PY{l+m+mf}{8.0}\PY{p}{,}\PY{l+m+mf}{15.5}\PY{p}{,}\PY{l+m+mf}{8.0}\PY{p}{,}\PY{l+m+mf}{9.0}\PY{p}{,}\PY{l+m+mf}{6.0}\PY{p}{,}\PY{l+m+mf}{10.0}\PY{p}{,}\PY{l+m+mf}{12.0}\PY{p}{,}\PY{l+m+mf}{12.5}\PY{p}{,}\PY{l+m+mf}{12.0}\PY{p}{]}

\PY{c+c1}{\PYZsh{} Create a 2D array (an array of arrays)}
\PY{n}{student\PYZus{}data} \PY{o}{=} \PY{n}{np}\PY{o}{.}\PY{n}{array}\PY{p}{(}\PY{p}{[}\PY{n}{study\PYZus{}hours}\PY{p}{,} \PY{n}{grades}\PY{p}{]}\PY{p}{)}

\PY{c+c1}{\PYZsh{} display the array}
\PY{n}{student\PYZus{}data}
\end{Verbatim}
\end{tcolorbox}

            \begin{tcolorbox}[breakable, size=fbox, boxrule=.5pt, pad at break*=1mm, opacityfill=0]
\prompt{Out}{outcolor}{33}{\boxspacing}
\begin{Verbatim}[commandchars=\\\{\}]
array([[10.  , 11.5 ,  9.  , 16.  ,  9.25,  1.  , 11.5 ,  9.  ,  8.5 ,
        14.5 , 15.5 , 13.75,  9.  ,  8.  , 15.5 ,  8.  ,  9.  ,  6.  ,
        10.  , 12.  , 12.5 , 12.  ],
       [50.  , 50.  , 47.  , 97.  , 49.  ,  3.  , 53.  , 42.  , 26.  ,
        74.  , 82.  , 62.  , 37.  , 15.  , 70.  , 27.  , 36.  , 35.  ,
        48.  , 52.  , 63.  , 64.  ]])
\end{Verbatim}
\end{tcolorbox}
        
    Now the data consists of a two-dimensional array; an array of arrays.
Let's look at its shape.

    \begin{tcolorbox}[breakable, size=fbox, boxrule=1pt, pad at break*=1mm,colback=cellbackground, colframe=cellborder]
\prompt{In}{incolor}{34}{\boxspacing}
\begin{Verbatim}[commandchars=\\\{\}]
\PY{c+c1}{\PYZsh{} Show shape of 2D array}
\PY{n}{student\PYZus{}data}\PY{o}{.}\PY{n}{shape}
\end{Verbatim}
\end{tcolorbox}

            \begin{tcolorbox}[breakable, size=fbox, boxrule=.5pt, pad at break*=1mm, opacityfill=0]
\prompt{Out}{outcolor}{34}{\boxspacing}
\begin{Verbatim}[commandchars=\\\{\}]
(2, 22)
\end{Verbatim}
\end{tcolorbox}
        
    The \textbf{student\_data} array contains two elements, each of which is
an array containing 22 elements.

To navigate this structure, you need to specify the position of each
element in the hierarchy. So to find the first value in the first array
(which contains the study hours data), you can use the following code.

    \begin{tcolorbox}[breakable, size=fbox, boxrule=1pt, pad at break*=1mm,colback=cellbackground, colframe=cellborder]
\prompt{In}{incolor}{35}{\boxspacing}
\begin{Verbatim}[commandchars=\\\{\}]
\PY{c+c1}{\PYZsh{} Show the first element of the first element}
\PY{n}{student\PYZus{}data}\PY{p}{[}\PY{l+m+mi}{0}\PY{p}{]}\PY{p}{[}\PY{l+m+mi}{0}\PY{p}{]}
\end{Verbatim}
\end{tcolorbox}

            \begin{tcolorbox}[breakable, size=fbox, boxrule=.5pt, pad at break*=1mm, opacityfill=0]
\prompt{Out}{outcolor}{35}{\boxspacing}
\begin{Verbatim}[commandchars=\\\{\}]
10.0
\end{Verbatim}
\end{tcolorbox}
        
    Now you have a multidimensional array containing both the student's
study time and grade information, which you can use to compare study
time to a student's grade.

    \begin{tcolorbox}[breakable, size=fbox, boxrule=1pt, pad at break*=1mm,colback=cellbackground, colframe=cellborder]
\prompt{In}{incolor}{36}{\boxspacing}
\begin{Verbatim}[commandchars=\\\{\}]
\PY{c+c1}{\PYZsh{} Get the mean value of each sub\PYZhy{}array}
\PY{n}{avg\PYZus{}study} \PY{o}{=} \PY{n}{student\PYZus{}data}\PY{p}{[}\PY{l+m+mi}{0}\PY{p}{]}\PY{o}{.}\PY{n}{mean}\PY{p}{(}\PY{p}{)}
\PY{n}{avg\PYZus{}grade} \PY{o}{=} \PY{n}{student\PYZus{}data}\PY{p}{[}\PY{l+m+mi}{1}\PY{p}{]}\PY{o}{.}\PY{n}{mean}\PY{p}{(}\PY{p}{)}

\PY{n+nb}{print}\PY{p}{(}\PY{l+s+s1}{\PYZsq{}}\PY{l+s+s1}{Average study hours: }\PY{l+s+si}{\PYZob{}:.2f\PYZcb{}}\PY{l+s+se}{\PYZbs{}n}\PY{l+s+s1}{Average grade: }\PY{l+s+si}{\PYZob{}:.2f\PYZcb{}}\PY{l+s+s1}{\PYZsq{}}\PY{o}{.}\PY{n}{format}\PY{p}{(}\PY{n}{avg\PYZus{}study}\PY{p}{,} \PY{n}{avg\PYZus{}grade}\PY{p}{)}\PY{p}{)}
\end{Verbatim}
\end{tcolorbox}

    \begin{Verbatim}[commandchars=\\\{\}]
Average study hours: 10.52
Average grade: 49.18
    \end{Verbatim}

    \hypertarget{exploring-tabular-data-with-pandas}{%
\subsection{Exploring tabular data with
Pandas}\label{exploring-tabular-data-with-pandas}}

NumPy provides a lot of the functionality and tools you need to work
with numbers, such as arrays of numeric values. However, when you start
to deal with two-dimensional tables of data, the \textbf{Pandas} package
offers a more convenient structure to work with: the \textbf{DataFrame}.

Run the following cell to import the Pandas library and create a
DataFrame with three columns. The first column is a list of student
names, and the second and third columns are the NumPy arrays containing
the study time and grade data.

    \begin{tcolorbox}[breakable, size=fbox, boxrule=1pt, pad at break*=1mm,colback=cellbackground, colframe=cellborder]
\prompt{In}{incolor}{37}{\boxspacing}
\begin{Verbatim}[commandchars=\\\{\}]
\PY{k+kn}{import} \PY{n+nn}{pandas} \PY{k}{as} \PY{n+nn}{pd}

\PY{n}{df\PYZus{}students} \PY{o}{=} \PY{n}{pd}\PY{o}{.}\PY{n}{DataFrame}\PY{p}{(}\PY{p}{\PYZob{}}\PY{l+s+s1}{\PYZsq{}}\PY{l+s+s1}{Name}\PY{l+s+s1}{\PYZsq{}}\PY{p}{:} \PY{p}{[}\PY{l+s+s1}{\PYZsq{}}\PY{l+s+s1}{Dan}\PY{l+s+s1}{\PYZsq{}}\PY{p}{,} \PY{l+s+s1}{\PYZsq{}}\PY{l+s+s1}{Joann}\PY{l+s+s1}{\PYZsq{}}\PY{p}{,} \PY{l+s+s1}{\PYZsq{}}\PY{l+s+s1}{Pedro}\PY{l+s+s1}{\PYZsq{}}\PY{p}{,} \PY{l+s+s1}{\PYZsq{}}\PY{l+s+s1}{Rosie}\PY{l+s+s1}{\PYZsq{}}\PY{p}{,} \PY{l+s+s1}{\PYZsq{}}\PY{l+s+s1}{Ethan}\PY{l+s+s1}{\PYZsq{}}\PY{p}{,} \PY{l+s+s1}{\PYZsq{}}\PY{l+s+s1}{Vicky}\PY{l+s+s1}{\PYZsq{}}\PY{p}{,} \PY{l+s+s1}{\PYZsq{}}\PY{l+s+s1}{Frederic}\PY{l+s+s1}{\PYZsq{}}\PY{p}{,} \PY{l+s+s1}{\PYZsq{}}\PY{l+s+s1}{Jimmie}\PY{l+s+s1}{\PYZsq{}}\PY{p}{,} 
                                     \PY{l+s+s1}{\PYZsq{}}\PY{l+s+s1}{Rhonda}\PY{l+s+s1}{\PYZsq{}}\PY{p}{,} \PY{l+s+s1}{\PYZsq{}}\PY{l+s+s1}{Giovanni}\PY{l+s+s1}{\PYZsq{}}\PY{p}{,} \PY{l+s+s1}{\PYZsq{}}\PY{l+s+s1}{Francesca}\PY{l+s+s1}{\PYZsq{}}\PY{p}{,} \PY{l+s+s1}{\PYZsq{}}\PY{l+s+s1}{Rajab}\PY{l+s+s1}{\PYZsq{}}\PY{p}{,} \PY{l+s+s1}{\PYZsq{}}\PY{l+s+s1}{Naiyana}\PY{l+s+s1}{\PYZsq{}}\PY{p}{,} \PY{l+s+s1}{\PYZsq{}}\PY{l+s+s1}{Kian}\PY{l+s+s1}{\PYZsq{}}\PY{p}{,} \PY{l+s+s1}{\PYZsq{}}\PY{l+s+s1}{Jenny}\PY{l+s+s1}{\PYZsq{}}\PY{p}{,}
                                     \PY{l+s+s1}{\PYZsq{}}\PY{l+s+s1}{Jakeem}\PY{l+s+s1}{\PYZsq{}}\PY{p}{,}\PY{l+s+s1}{\PYZsq{}}\PY{l+s+s1}{Helena}\PY{l+s+s1}{\PYZsq{}}\PY{p}{,}\PY{l+s+s1}{\PYZsq{}}\PY{l+s+s1}{Ismat}\PY{l+s+s1}{\PYZsq{}}\PY{p}{,}\PY{l+s+s1}{\PYZsq{}}\PY{l+s+s1}{Anila}\PY{l+s+s1}{\PYZsq{}}\PY{p}{,}\PY{l+s+s1}{\PYZsq{}}\PY{l+s+s1}{Skye}\PY{l+s+s1}{\PYZsq{}}\PY{p}{,}\PY{l+s+s1}{\PYZsq{}}\PY{l+s+s1}{Daniel}\PY{l+s+s1}{\PYZsq{}}\PY{p}{,}\PY{l+s+s1}{\PYZsq{}}\PY{l+s+s1}{Aisha}\PY{l+s+s1}{\PYZsq{}}\PY{p}{]}\PY{p}{,}
                            \PY{l+s+s1}{\PYZsq{}}\PY{l+s+s1}{StudyHours}\PY{l+s+s1}{\PYZsq{}}\PY{p}{:}\PY{n}{student\PYZus{}data}\PY{p}{[}\PY{l+m+mi}{0}\PY{p}{]}\PY{p}{,}
                            \PY{l+s+s1}{\PYZsq{}}\PY{l+s+s1}{Grade}\PY{l+s+s1}{\PYZsq{}}\PY{p}{:}\PY{n}{student\PYZus{}data}\PY{p}{[}\PY{l+m+mi}{1}\PY{p}{]}\PY{p}{\PYZcb{}}\PY{p}{)}

\PY{n}{df\PYZus{}students} 
\end{Verbatim}
\end{tcolorbox}

            \begin{tcolorbox}[breakable, size=fbox, boxrule=.5pt, pad at break*=1mm, opacityfill=0]
\prompt{Out}{outcolor}{37}{\boxspacing}
\begin{Verbatim}[commandchars=\\\{\}]
         Name  StudyHours  Grade
0         Dan       10.00   50.0
1       Joann       11.50   50.0
2       Pedro        9.00   47.0
3       Rosie       16.00   97.0
4       Ethan        9.25   49.0
5       Vicky        1.00    3.0
6    Frederic       11.50   53.0
7      Jimmie        9.00   42.0
8      Rhonda        8.50   26.0
9    Giovanni       14.50   74.0
10  Francesca       15.50   82.0
11      Rajab       13.75   62.0
12    Naiyana        9.00   37.0
13       Kian        8.00   15.0
14      Jenny       15.50   70.0
15     Jakeem        8.00   27.0
16     Helena        9.00   36.0
17      Ismat        6.00   35.0
18      Anila       10.00   48.0
19       Skye       12.00   52.0
20     Daniel       12.50   63.0
21      Aisha       12.00   64.0
\end{Verbatim}
\end{tcolorbox}
        
    Note that in addition to the columns you specified, the DataFrame
includes an \emph{index} to uniquely identify each row. We could've
specified the index explicitly and assigned any kind of appropriate
value (for example, an email address). However, because we didn't
specify an index, one has been created with a unique integer value for
each row.

\hypertarget{finding-and-filtering-data-in-a-dataframe}{%
\subsubsection{Finding and filtering data in a
DataFrame}\label{finding-and-filtering-data-in-a-dataframe}}

You can use the DataFrame's \textbf{loc} method to retrieve data for a
specific index value, like this.

    \begin{tcolorbox}[breakable, size=fbox, boxrule=1pt, pad at break*=1mm,colback=cellbackground, colframe=cellborder]
\prompt{In}{incolor}{38}{\boxspacing}
\begin{Verbatim}[commandchars=\\\{\}]
\PY{c+c1}{\PYZsh{} Get the data for index value 5}
\PY{n}{df\PYZus{}students}\PY{o}{.}\PY{n}{loc}\PY{p}{[}\PY{l+m+mi}{5}\PY{p}{]}
\end{Verbatim}
\end{tcolorbox}

            \begin{tcolorbox}[breakable, size=fbox, boxrule=.5pt, pad at break*=1mm, opacityfill=0]
\prompt{Out}{outcolor}{38}{\boxspacing}
\begin{Verbatim}[commandchars=\\\{\}]
Name          Vicky
StudyHours        1
Grade             3
Name: 5, dtype: object
\end{Verbatim}
\end{tcolorbox}
        
    You can also get the data at a range of index values, like this:

    \begin{tcolorbox}[breakable, size=fbox, boxrule=1pt, pad at break*=1mm,colback=cellbackground, colframe=cellborder]
\prompt{In}{incolor}{39}{\boxspacing}
\begin{Verbatim}[commandchars=\\\{\}]
\PY{c+c1}{\PYZsh{} Get the rows with index values from 0 to 5}
\PY{n}{df\PYZus{}students}\PY{o}{.}\PY{n}{loc}\PY{p}{[}\PY{l+m+mi}{0}\PY{p}{:}\PY{l+m+mi}{5}\PY{p}{]}
\end{Verbatim}
\end{tcolorbox}

            \begin{tcolorbox}[breakable, size=fbox, boxrule=.5pt, pad at break*=1mm, opacityfill=0]
\prompt{Out}{outcolor}{39}{\boxspacing}
\begin{Verbatim}[commandchars=\\\{\}]
    Name  StudyHours  Grade
0    Dan       10.00   50.0
1  Joann       11.50   50.0
2  Pedro        9.00   47.0
3  Rosie       16.00   97.0
4  Ethan        9.25   49.0
5  Vicky        1.00    3.0
\end{Verbatim}
\end{tcolorbox}
        
    In addition to being able to use the \textbf{loc} method to find rows
based on the index, you can use the \textbf{iloc} method to find rows
based on their ordinal position in the DataFrame (regardless of the
index):

    \begin{tcolorbox}[breakable, size=fbox, boxrule=1pt, pad at break*=1mm,colback=cellbackground, colframe=cellborder]
\prompt{In}{incolor}{40}{\boxspacing}
\begin{Verbatim}[commandchars=\\\{\}]
\PY{c+c1}{\PYZsh{} Get data in the first five rows}
\PY{n}{df\PYZus{}students}\PY{o}{.}\PY{n}{iloc}\PY{p}{[}\PY{l+m+mi}{0}\PY{p}{:}\PY{l+m+mi}{5}\PY{p}{]}
\end{Verbatim}
\end{tcolorbox}

            \begin{tcolorbox}[breakable, size=fbox, boxrule=.5pt, pad at break*=1mm, opacityfill=0]
\prompt{Out}{outcolor}{40}{\boxspacing}
\begin{Verbatim}[commandchars=\\\{\}]
    Name  StudyHours  Grade
0    Dan       10.00   50.0
1  Joann       11.50   50.0
2  Pedro        9.00   47.0
3  Rosie       16.00   97.0
4  Ethan        9.25   49.0
\end{Verbatim}
\end{tcolorbox}
        
    Look carefully at the \texttt{iloc{[}0:5{]}} results and compare them to
the \texttt{loc{[}0:5{]}} results you obtained previously. Can you spot
the difference?

The \textbf{loc} method returned rows with index \emph{label} in the
list of values from \emph{0} to \emph{5}, which includes \emph{0},
\emph{1}, \emph{2}, \emph{3}, \emph{4}, and \emph{5} (six rows).
However, the \textbf{iloc} method returns the rows in the
\emph{positions} included in the range 0 to 5. Since integer ranges
don't include the upper-bound value, this includes positions \emph{0},
\emph{1}, \emph{2}, \emph{3}, and \emph{4} (five rows).

\textbf{iloc} identifies data values in a DataFrame by \emph{position},
which extends beyond rows to columns. So, for example, you can use it to
find the values for the columns in positions 1 and 2 in row 0, like
this:

    \begin{tcolorbox}[breakable, size=fbox, boxrule=1pt, pad at break*=1mm,colback=cellbackground, colframe=cellborder]
\prompt{In}{incolor}{41}{\boxspacing}
\begin{Verbatim}[commandchars=\\\{\}]
\PY{n}{df\PYZus{}students}\PY{o}{.}\PY{n}{iloc}\PY{p}{[}\PY{l+m+mi}{0}\PY{p}{,}\PY{p}{[}\PY{l+m+mi}{1}\PY{p}{,}\PY{l+m+mi}{2}\PY{p}{]}\PY{p}{]}
\end{Verbatim}
\end{tcolorbox}

            \begin{tcolorbox}[breakable, size=fbox, boxrule=.5pt, pad at break*=1mm, opacityfill=0]
\prompt{Out}{outcolor}{41}{\boxspacing}
\begin{Verbatim}[commandchars=\\\{\}]
StudyHours    10
Grade         50
Name: 0, dtype: object
\end{Verbatim}
\end{tcolorbox}
        
    Let's return to the \textbf{loc} method and see how it works with
columns. Remember that you use \textbf{loc} to locate data items based
on index values rather than positions. In the absence of an explicit
index column, the rows in our DataFrame are indexed as integer values,
but the columns are identified by name:

    \begin{tcolorbox}[breakable, size=fbox, boxrule=1pt, pad at break*=1mm,colback=cellbackground, colframe=cellborder]
\prompt{In}{incolor}{42}{\boxspacing}
\begin{Verbatim}[commandchars=\\\{\}]
\PY{n}{df\PYZus{}students}\PY{o}{.}\PY{n}{loc}\PY{p}{[}\PY{l+m+mi}{0}\PY{p}{,}\PY{l+s+s1}{\PYZsq{}}\PY{l+s+s1}{Grade}\PY{l+s+s1}{\PYZsq{}}\PY{p}{]}
\end{Verbatim}
\end{tcolorbox}

            \begin{tcolorbox}[breakable, size=fbox, boxrule=.5pt, pad at break*=1mm, opacityfill=0]
\prompt{Out}{outcolor}{42}{\boxspacing}
\begin{Verbatim}[commandchars=\\\{\}]
50.0
\end{Verbatim}
\end{tcolorbox}
        
    Here's another useful trick. You can use the \textbf{loc} method to find
indexed rows based on a filtering expression that references named
columns other than the index, like this:

    \begin{tcolorbox}[breakable, size=fbox, boxrule=1pt, pad at break*=1mm,colback=cellbackground, colframe=cellborder]
\prompt{In}{incolor}{43}{\boxspacing}
\begin{Verbatim}[commandchars=\\\{\}]
\PY{n}{df\PYZus{}students}\PY{o}{.}\PY{n}{loc}\PY{p}{[}\PY{n}{df\PYZus{}students}\PY{p}{[}\PY{l+s+s1}{\PYZsq{}}\PY{l+s+s1}{Name}\PY{l+s+s1}{\PYZsq{}}\PY{p}{]}\PY{o}{==}\PY{l+s+s1}{\PYZsq{}}\PY{l+s+s1}{Aisha}\PY{l+s+s1}{\PYZsq{}}\PY{p}{]}
\end{Verbatim}
\end{tcolorbox}

            \begin{tcolorbox}[breakable, size=fbox, boxrule=.5pt, pad at break*=1mm, opacityfill=0]
\prompt{Out}{outcolor}{43}{\boxspacing}
\begin{Verbatim}[commandchars=\\\{\}]
     Name  StudyHours  Grade
21  Aisha        12.0   64.0
\end{Verbatim}
\end{tcolorbox}
        
    Actually, you don't need to explicitly use the \textbf{loc} method to do
this. You can simply apply a DataFrame filtering expression, like this:

    \begin{tcolorbox}[breakable, size=fbox, boxrule=1pt, pad at break*=1mm,colback=cellbackground, colframe=cellborder]
\prompt{In}{incolor}{44}{\boxspacing}
\begin{Verbatim}[commandchars=\\\{\}]
\PY{n}{df\PYZus{}students}\PY{p}{[}\PY{n}{df\PYZus{}students}\PY{p}{[}\PY{l+s+s1}{\PYZsq{}}\PY{l+s+s1}{Name}\PY{l+s+s1}{\PYZsq{}}\PY{p}{]}\PY{o}{==}\PY{l+s+s1}{\PYZsq{}}\PY{l+s+s1}{Aisha}\PY{l+s+s1}{\PYZsq{}}\PY{p}{]}
\end{Verbatim}
\end{tcolorbox}

            \begin{tcolorbox}[breakable, size=fbox, boxrule=.5pt, pad at break*=1mm, opacityfill=0]
\prompt{Out}{outcolor}{44}{\boxspacing}
\begin{Verbatim}[commandchars=\\\{\}]
     Name  StudyHours  Grade
21  Aisha        12.0   64.0
\end{Verbatim}
\end{tcolorbox}
        
    And for good measure, you can achieve the same results by using the
DataFrame's \textbf{query} method, like this:

    \begin{tcolorbox}[breakable, size=fbox, boxrule=1pt, pad at break*=1mm,colback=cellbackground, colframe=cellborder]
\prompt{In}{incolor}{45}{\boxspacing}
\begin{Verbatim}[commandchars=\\\{\}]
\PY{n}{df\PYZus{}students}\PY{o}{.}\PY{n}{query}\PY{p}{(}\PY{l+s+s1}{\PYZsq{}}\PY{l+s+s1}{Name==}\PY{l+s+s1}{\PYZdq{}}\PY{l+s+s1}{Aisha}\PY{l+s+s1}{\PYZdq{}}\PY{l+s+s1}{\PYZsq{}}\PY{p}{)}
\end{Verbatim}
\end{tcolorbox}

            \begin{tcolorbox}[breakable, size=fbox, boxrule=.5pt, pad at break*=1mm, opacityfill=0]
\prompt{Out}{outcolor}{45}{\boxspacing}
\begin{Verbatim}[commandchars=\\\{\}]
     Name  StudyHours  Grade
21  Aisha        12.0   64.0
\end{Verbatim}
\end{tcolorbox}
        
    The three previous examples underline a confusing truth about working
with Pandas. Often, there are multiple ways to achieve the same results.
Another example of this is the way you refer to a DataFrame column name.
You can specify the column name as a named index value (as in the
\texttt{df\_students{[}\textquotesingle{}Name\textquotesingle{}{]}}
examples we've seen so far), or you can use the column as a property of
the DataFrame, like this:

    \begin{tcolorbox}[breakable, size=fbox, boxrule=1pt, pad at break*=1mm,colback=cellbackground, colframe=cellborder]
\prompt{In}{incolor}{46}{\boxspacing}
\begin{Verbatim}[commandchars=\\\{\}]
\PY{n}{df\PYZus{}students}\PY{p}{[}\PY{n}{df\PYZus{}students}\PY{o}{.}\PY{n}{Name} \PY{o}{==} \PY{l+s+s1}{\PYZsq{}}\PY{l+s+s1}{Aisha}\PY{l+s+s1}{\PYZsq{}}\PY{p}{]}
\end{Verbatim}
\end{tcolorbox}

            \begin{tcolorbox}[breakable, size=fbox, boxrule=.5pt, pad at break*=1mm, opacityfill=0]
\prompt{Out}{outcolor}{46}{\boxspacing}
\begin{Verbatim}[commandchars=\\\{\}]
     Name  StudyHours  Grade
21  Aisha        12.0   64.0
\end{Verbatim}
\end{tcolorbox}
        
    \hypertarget{loading-a-dataframe-from-a-file}{%
\subsubsection{Loading a DataFrame from a
file}\label{loading-a-dataframe-from-a-file}}

We constructed the DataFrame from some existing arrays. However, in many
real-world scenarios, data is loaded from sources such as files. Let's
replace the student grades DataFrame with the contents of a text file.

    \begin{tcolorbox}[breakable, size=fbox, boxrule=1pt, pad at break*=1mm,colback=cellbackground, colframe=cellborder]
\prompt{In}{incolor}{47}{\boxspacing}
\begin{Verbatim}[commandchars=\\\{\}]
\PY{o}{!}wget\PY{+w}{ }https://raw.githubusercontent.com/MicrosoftDocs/mslearn\PYZhy{}introduction\PYZhy{}to\PYZhy{}machine\PYZhy{}learning/main/Data/ml\PYZhy{}basics/grades.csv
\PY{n}{df\PYZus{}students} \PY{o}{=} \PY{n}{pd}\PY{o}{.}\PY{n}{read\PYZus{}csv}\PY{p}{(}\PY{l+s+s1}{\PYZsq{}}\PY{l+s+s1}{grades.csv}\PY{l+s+s1}{\PYZsq{}}\PY{p}{,}\PY{n}{delimiter}\PY{o}{=}\PY{l+s+s1}{\PYZsq{}}\PY{l+s+s1}{,}\PY{l+s+s1}{\PYZsq{}}\PY{p}{,}\PY{n}{header}\PY{o}{=}\PY{l+s+s1}{\PYZsq{}}\PY{l+s+s1}{infer}\PY{l+s+s1}{\PYZsq{}}\PY{p}{)}
\PY{n}{df\PYZus{}students}\PY{o}{.}\PY{n}{head}\PY{p}{(}\PY{p}{)}
\end{Verbatim}
\end{tcolorbox}

    \begin{Verbatim}[commandchars=\\\{\}]
--2023-08-07 19:23:07--
https://raw.githubusercontent.com/MicrosoftDocs/mslearn-introduction-to-machine-
learning/main/Data/ml-basics/grades.csv
Resolving raw.githubusercontent.com (raw.githubusercontent.com){\ldots}
185.199.109.133, 185.199.110.133, 185.199.111.133, {\ldots}
Connecting to raw.githubusercontent.com
(raw.githubusercontent.com)|185.199.109.133|:443{\ldots} connected.
HTTP request sent, awaiting response{\ldots} 200 OK
Length: 322 [text/plain]
Saving to: ‘grades.csv’

grades.csv          100\%[===================>]     322  --.-KB/s    in 0s

2023-08-07 19:23:07 (17.9 MB/s) - ‘grades.csv’ saved [322/322]

    \end{Verbatim}

            \begin{tcolorbox}[breakable, size=fbox, boxrule=.5pt, pad at break*=1mm, opacityfill=0]
\prompt{Out}{outcolor}{47}{\boxspacing}
\begin{Verbatim}[commandchars=\\\{\}]
    Name  StudyHours  Grade
0    Dan       10.00   50.0
1  Joann       11.50   50.0
2  Pedro        9.00   47.0
3  Rosie       16.00   97.0
4  Ethan        9.25   49.0
\end{Verbatim}
\end{tcolorbox}
        
    The DataFrame's \textbf{read\_csv} method is used to load data from text
files. As you can see in the example code, you can specify options such
as the column delimiter and which row (if any) contains column headers.
(In this case, the delimiter is a comma and the first row contains the
column names. These are the default settings, so we could've omitted the
parameters.)

\hypertarget{handling-missing-values}{%
\subsubsection{Handling missing values}\label{handling-missing-values}}

One of the most common issues data scientists need to deal with is
incomplete or missing data. So how would we know that the DataFrame
contains missing values? You can use the \textbf{isnull} method to
identify which individual values are null, like this:

    \begin{tcolorbox}[breakable, size=fbox, boxrule=1pt, pad at break*=1mm,colback=cellbackground, colframe=cellborder]
\prompt{In}{incolor}{48}{\boxspacing}
\begin{Verbatim}[commandchars=\\\{\}]
\PY{n}{df\PYZus{}students}\PY{o}{.}\PY{n}{isnull}\PY{p}{(}\PY{p}{)}
\end{Verbatim}
\end{tcolorbox}

            \begin{tcolorbox}[breakable, size=fbox, boxrule=.5pt, pad at break*=1mm, opacityfill=0]
\prompt{Out}{outcolor}{48}{\boxspacing}
\begin{Verbatim}[commandchars=\\\{\}]
     Name  StudyHours  Grade
0   False       False  False
1   False       False  False
2   False       False  False
3   False       False  False
4   False       False  False
5   False       False  False
6   False       False  False
7   False       False  False
8   False       False  False
9   False       False  False
10  False       False  False
11  False       False  False
12  False       False  False
13  False       False  False
14  False       False  False
15  False       False  False
16  False       False  False
17  False       False  False
18  False       False  False
19  False       False  False
20  False       False  False
21  False       False  False
22  False       False   True
23  False        True   True
\end{Verbatim}
\end{tcolorbox}
        
    Of course, with a larger DataFrame, it would be inefficient to review
all of the rows and columns individually, so we can get the sum of
missing values for each column like this:

    \begin{tcolorbox}[breakable, size=fbox, boxrule=1pt, pad at break*=1mm,colback=cellbackground, colframe=cellborder]
\prompt{In}{incolor}{49}{\boxspacing}
\begin{Verbatim}[commandchars=\\\{\}]
\PY{n}{df\PYZus{}students}\PY{o}{.}\PY{n}{isnull}\PY{p}{(}\PY{p}{)}\PY{o}{.}\PY{n}{sum}\PY{p}{(}\PY{p}{)}
\end{Verbatim}
\end{tcolorbox}

            \begin{tcolorbox}[breakable, size=fbox, boxrule=.5pt, pad at break*=1mm, opacityfill=0]
\prompt{Out}{outcolor}{49}{\boxspacing}
\begin{Verbatim}[commandchars=\\\{\}]
Name          0
StudyHours    1
Grade         2
dtype: int64
\end{Verbatim}
\end{tcolorbox}
        
    So now we know that there's one missing \textbf{StudyHours} value and
two missing \textbf{Grade} values.

To see them in context, we can filter the DataFrame to include only rows
where any of the columns (axis 1 of the DataFrame) are null.

    \begin{tcolorbox}[breakable, size=fbox, boxrule=1pt, pad at break*=1mm,colback=cellbackground, colframe=cellborder]
\prompt{In}{incolor}{50}{\boxspacing}
\begin{Verbatim}[commandchars=\\\{\}]
\PY{n}{df\PYZus{}students}\PY{p}{[}\PY{n}{df\PYZus{}students}\PY{o}{.}\PY{n}{isnull}\PY{p}{(}\PY{p}{)}\PY{o}{.}\PY{n}{any}\PY{p}{(}\PY{n}{axis}\PY{o}{=}\PY{l+m+mi}{1}\PY{p}{)}\PY{p}{]}
\end{Verbatim}
\end{tcolorbox}

            \begin{tcolorbox}[breakable, size=fbox, boxrule=.5pt, pad at break*=1mm, opacityfill=0]
\prompt{Out}{outcolor}{50}{\boxspacing}
\begin{Verbatim}[commandchars=\\\{\}]
    Name  StudyHours  Grade
22  Bill         8.0    NaN
23   Ted         NaN    NaN
\end{Verbatim}
\end{tcolorbox}
        
    When the DataFrame is retrieved, the missing numeric values show up as
\textbf{NaN} (\emph{not a number}).

So now that we've found the null values, what can we do about them?

One common approach is to \emph{impute} replacement values. For example,
if the number of study hours is missing, we could just assume that the
student studied for an average amount of time and replace the missing
value with the mean study hours. To do this, we can use the
\textbf{fillna} method, like this:

    \begin{tcolorbox}[breakable, size=fbox, boxrule=1pt, pad at break*=1mm,colback=cellbackground, colframe=cellborder]
\prompt{In}{incolor}{51}{\boxspacing}
\begin{Verbatim}[commandchars=\\\{\}]
\PY{n}{df\PYZus{}students}\PY{o}{.}\PY{n}{StudyHours} \PY{o}{=} \PY{n}{df\PYZus{}students}\PY{o}{.}\PY{n}{StudyHours}\PY{o}{.}\PY{n}{fillna}\PY{p}{(}\PY{n}{df\PYZus{}students}\PY{o}{.}\PY{n}{StudyHours}\PY{o}{.}\PY{n}{mean}\PY{p}{(}\PY{p}{)}\PY{p}{)}
\PY{n}{df\PYZus{}students}
\end{Verbatim}
\end{tcolorbox}

            \begin{tcolorbox}[breakable, size=fbox, boxrule=.5pt, pad at break*=1mm, opacityfill=0]
\prompt{Out}{outcolor}{51}{\boxspacing}
\begin{Verbatim}[commandchars=\\\{\}]
         Name  StudyHours  Grade
0         Dan   10.000000   50.0
1       Joann   11.500000   50.0
2       Pedro    9.000000   47.0
3       Rosie   16.000000   97.0
4       Ethan    9.250000   49.0
5       Vicky    1.000000    3.0
6    Frederic   11.500000   53.0
7      Jimmie    9.000000   42.0
8      Rhonda    8.500000   26.0
9    Giovanni   14.500000   74.0
10  Francesca   15.500000   82.0
11      Rajab   13.750000   62.0
12    Naiyana    9.000000   37.0
13       Kian    8.000000   15.0
14      Jenny   15.500000   70.0
15     Jakeem    8.000000   27.0
16     Helena    9.000000   36.0
17      Ismat    6.000000   35.0
18      Anila   10.000000   48.0
19       Skye   12.000000   52.0
20     Daniel   12.500000   63.0
21      Aisha   12.000000   64.0
22       Bill    8.000000    NaN
23        Ted   10.413043    NaN
\end{Verbatim}
\end{tcolorbox}
        
    Alternatively, it might be important to ensure that you only use data
you know to be absolutely correct. In this case, you can drop rows or
columns that contain null values by using the \textbf{dropna} method.
For example, we'll remove rows (axis 0 of the DataFrame) where any of
the columns contain null values:

    \begin{tcolorbox}[breakable, size=fbox, boxrule=1pt, pad at break*=1mm,colback=cellbackground, colframe=cellborder]
\prompt{In}{incolor}{52}{\boxspacing}
\begin{Verbatim}[commandchars=\\\{\}]
\PY{n}{df\PYZus{}students} \PY{o}{=} \PY{n}{df\PYZus{}students}\PY{o}{.}\PY{n}{dropna}\PY{p}{(}\PY{n}{axis}\PY{o}{=}\PY{l+m+mi}{0}\PY{p}{,} \PY{n}{how}\PY{o}{=}\PY{l+s+s1}{\PYZsq{}}\PY{l+s+s1}{any}\PY{l+s+s1}{\PYZsq{}}\PY{p}{)}
\PY{n}{df\PYZus{}students}
\end{Verbatim}
\end{tcolorbox}

            \begin{tcolorbox}[breakable, size=fbox, boxrule=.5pt, pad at break*=1mm, opacityfill=0]
\prompt{Out}{outcolor}{52}{\boxspacing}
\begin{Verbatim}[commandchars=\\\{\}]
         Name  StudyHours  Grade
0         Dan       10.00   50.0
1       Joann       11.50   50.0
2       Pedro        9.00   47.0
3       Rosie       16.00   97.0
4       Ethan        9.25   49.0
5       Vicky        1.00    3.0
6    Frederic       11.50   53.0
7      Jimmie        9.00   42.0
8      Rhonda        8.50   26.0
9    Giovanni       14.50   74.0
10  Francesca       15.50   82.0
11      Rajab       13.75   62.0
12    Naiyana        9.00   37.0
13       Kian        8.00   15.0
14      Jenny       15.50   70.0
15     Jakeem        8.00   27.0
16     Helena        9.00   36.0
17      Ismat        6.00   35.0
18      Anila       10.00   48.0
19       Skye       12.00   52.0
20     Daniel       12.50   63.0
21      Aisha       12.00   64.0
\end{Verbatim}
\end{tcolorbox}
        
    \hypertarget{explore-data-in-the-dataframe}{%
\subsubsection{Explore data in the
DataFrame}\label{explore-data-in-the-dataframe}}

Now that we've cleaned up the missing values, we're ready to explore the
data in the DataFrame. Let's start by comparing the mean study hours and
grades.

    \begin{tcolorbox}[breakable, size=fbox, boxrule=1pt, pad at break*=1mm,colback=cellbackground, colframe=cellborder]
\prompt{In}{incolor}{53}{\boxspacing}
\begin{Verbatim}[commandchars=\\\{\}]
\PY{c+c1}{\PYZsh{} Get the mean study hours using to column name as an index}
\PY{n}{mean\PYZus{}study} \PY{o}{=} \PY{n}{df\PYZus{}students}\PY{p}{[}\PY{l+s+s1}{\PYZsq{}}\PY{l+s+s1}{StudyHours}\PY{l+s+s1}{\PYZsq{}}\PY{p}{]}\PY{o}{.}\PY{n}{mean}\PY{p}{(}\PY{p}{)}

\PY{c+c1}{\PYZsh{} Get the mean grade using the column name as a property (just to make the point!)}
\PY{n}{mean\PYZus{}grade} \PY{o}{=} \PY{n}{df\PYZus{}students}\PY{o}{.}\PY{n}{Grade}\PY{o}{.}\PY{n}{mean}\PY{p}{(}\PY{p}{)}

\PY{c+c1}{\PYZsh{} Print the mean study hours and mean grade}
\PY{n+nb}{print}\PY{p}{(}\PY{l+s+s1}{\PYZsq{}}\PY{l+s+s1}{Average weekly study hours: }\PY{l+s+si}{\PYZob{}:.2f\PYZcb{}}\PY{l+s+se}{\PYZbs{}n}\PY{l+s+s1}{Average grade: }\PY{l+s+si}{\PYZob{}:.2f\PYZcb{}}\PY{l+s+s1}{\PYZsq{}}\PY{o}{.}\PY{n}{format}\PY{p}{(}\PY{n}{mean\PYZus{}study}\PY{p}{,} \PY{n}{mean\PYZus{}grade}\PY{p}{)}\PY{p}{)}
\end{Verbatim}
\end{tcolorbox}

    \begin{Verbatim}[commandchars=\\\{\}]
Average weekly study hours: 10.52
Average grade: 49.18
    \end{Verbatim}

    OK, let's filter the DataFrame to find only the students who studied for
more than the average amount of time.

    \begin{tcolorbox}[breakable, size=fbox, boxrule=1pt, pad at break*=1mm,colback=cellbackground, colframe=cellborder]
\prompt{In}{incolor}{54}{\boxspacing}
\begin{Verbatim}[commandchars=\\\{\}]
\PY{c+c1}{\PYZsh{} Get students who studied for the mean or more hours}
\PY{n}{df\PYZus{}students}\PY{p}{[}\PY{n}{df\PYZus{}students}\PY{o}{.}\PY{n}{StudyHours} \PY{o}{\PYZgt{}} \PY{n}{mean\PYZus{}study}\PY{p}{]}
\end{Verbatim}
\end{tcolorbox}

            \begin{tcolorbox}[breakable, size=fbox, boxrule=.5pt, pad at break*=1mm, opacityfill=0]
\prompt{Out}{outcolor}{54}{\boxspacing}
\begin{Verbatim}[commandchars=\\\{\}]
         Name  StudyHours  Grade
1       Joann       11.50   50.0
3       Rosie       16.00   97.0
6    Frederic       11.50   53.0
9    Giovanni       14.50   74.0
10  Francesca       15.50   82.0
11      Rajab       13.75   62.0
14      Jenny       15.50   70.0
19       Skye       12.00   52.0
20     Daniel       12.50   63.0
21      Aisha       12.00   64.0
\end{Verbatim}
\end{tcolorbox}
        
    Note that the filtered result is itself a DataFrame, so you can work
with its columns just like any other DataFrame.

For example, let's find the average grade for students who undertook
more than the average amount of study time.

    \begin{tcolorbox}[breakable, size=fbox, boxrule=1pt, pad at break*=1mm,colback=cellbackground, colframe=cellborder]
\prompt{In}{incolor}{55}{\boxspacing}
\begin{Verbatim}[commandchars=\\\{\}]
\PY{c+c1}{\PYZsh{} What was their mean grade?}
\PY{n}{df\PYZus{}students}\PY{p}{[}\PY{n}{df\PYZus{}students}\PY{o}{.}\PY{n}{StudyHours} \PY{o}{\PYZgt{}} \PY{n}{mean\PYZus{}study}\PY{p}{]}\PY{o}{.}\PY{n}{Grade}\PY{o}{.}\PY{n}{mean}\PY{p}{(}\PY{p}{)}
\end{Verbatim}
\end{tcolorbox}

            \begin{tcolorbox}[breakable, size=fbox, boxrule=.5pt, pad at break*=1mm, opacityfill=0]
\prompt{Out}{outcolor}{55}{\boxspacing}
\begin{Verbatim}[commandchars=\\\{\}]
66.7
\end{Verbatim}
\end{tcolorbox}
        
    Let's assume that the passing grade for the course is 60.

We can use that information to add a new column to the DataFrame that
indicates whether or not each student passed.

First, we'll create a Pandas \textbf{Series} containing the pass/fail
indicator (True or False), and then we'll concatenate that series as a
new column (axis 1) in the DataFrame.

    \begin{tcolorbox}[breakable, size=fbox, boxrule=1pt, pad at break*=1mm,colback=cellbackground, colframe=cellborder]
\prompt{In}{incolor}{56}{\boxspacing}
\begin{Verbatim}[commandchars=\\\{\}]
\PY{n}{passes}  \PY{o}{=} \PY{n}{pd}\PY{o}{.}\PY{n}{Series}\PY{p}{(}\PY{n}{df\PYZus{}students}\PY{p}{[}\PY{l+s+s1}{\PYZsq{}}\PY{l+s+s1}{Grade}\PY{l+s+s1}{\PYZsq{}}\PY{p}{]} \PY{o}{\PYZgt{}}\PY{o}{=} \PY{l+m+mi}{60}\PY{p}{)}
\PY{n}{df\PYZus{}students} \PY{o}{=} \PY{n}{pd}\PY{o}{.}\PY{n}{concat}\PY{p}{(}\PY{p}{[}\PY{n}{df\PYZus{}students}\PY{p}{,} \PY{n}{passes}\PY{o}{.}\PY{n}{rename}\PY{p}{(}\PY{l+s+s2}{\PYZdq{}}\PY{l+s+s2}{Pass}\PY{l+s+s2}{\PYZdq{}}\PY{p}{)}\PY{p}{]}\PY{p}{,} \PY{n}{axis}\PY{o}{=}\PY{l+m+mi}{1}\PY{p}{)}

\PY{n}{df\PYZus{}students}
\end{Verbatim}
\end{tcolorbox}

            \begin{tcolorbox}[breakable, size=fbox, boxrule=.5pt, pad at break*=1mm, opacityfill=0]
\prompt{Out}{outcolor}{56}{\boxspacing}
\begin{Verbatim}[commandchars=\\\{\}]
         Name  StudyHours  Grade   Pass
0         Dan       10.00   50.0  False
1       Joann       11.50   50.0  False
2       Pedro        9.00   47.0  False
3       Rosie       16.00   97.0   True
4       Ethan        9.25   49.0  False
5       Vicky        1.00    3.0  False
6    Frederic       11.50   53.0  False
7      Jimmie        9.00   42.0  False
8      Rhonda        8.50   26.0  False
9    Giovanni       14.50   74.0   True
10  Francesca       15.50   82.0   True
11      Rajab       13.75   62.0   True
12    Naiyana        9.00   37.0  False
13       Kian        8.00   15.0  False
14      Jenny       15.50   70.0   True
15     Jakeem        8.00   27.0  False
16     Helena        9.00   36.0  False
17      Ismat        6.00   35.0  False
18      Anila       10.00   48.0  False
19       Skye       12.00   52.0  False
20     Daniel       12.50   63.0   True
21      Aisha       12.00   64.0   True
\end{Verbatim}
\end{tcolorbox}
        
    DataFrames are designed for tabular data, and you can use them to
perform many of the same kinds of data-analytics operations you can do
in a relational database, such as grouping and aggregating tables of
data.

For example, you can use the \textbf{groupby} method to group the
student data into groups based on the \textbf{Pass} column you added
previously and to count the number of names in each group. In other
words, you can determine how many students passed and failed.

    \begin{tcolorbox}[breakable, size=fbox, boxrule=1pt, pad at break*=1mm,colback=cellbackground, colframe=cellborder]
\prompt{In}{incolor}{57}{\boxspacing}
\begin{Verbatim}[commandchars=\\\{\}]
\PY{n+nb}{print}\PY{p}{(}\PY{n}{df\PYZus{}students}\PY{o}{.}\PY{n}{groupby}\PY{p}{(}\PY{n}{df\PYZus{}students}\PY{o}{.}\PY{n}{Pass}\PY{p}{)}\PY{o}{.}\PY{n}{Name}\PY{o}{.}\PY{n}{count}\PY{p}{(}\PY{p}{)}\PY{p}{)}
\end{Verbatim}
\end{tcolorbox}

    \begin{Verbatim}[commandchars=\\\{\}]
Pass
False    15
True      7
Name: Name, dtype: int64
    \end{Verbatim}

    You can aggregate multiple fields in a group using any available
aggregation function. For example, you can find the mean study time and
grade for the groups of students who passed and failed the course.

    \begin{tcolorbox}[breakable, size=fbox, boxrule=1pt, pad at break*=1mm,colback=cellbackground, colframe=cellborder]
\prompt{In}{incolor}{58}{\boxspacing}
\begin{Verbatim}[commandchars=\\\{\}]
\PY{n+nb}{print}\PY{p}{(}\PY{n}{df\PYZus{}students}\PY{o}{.}\PY{n}{groupby}\PY{p}{(}\PY{n}{df\PYZus{}students}\PY{o}{.}\PY{n}{Pass}\PY{p}{)}\PY{p}{[}\PY{l+s+s1}{\PYZsq{}}\PY{l+s+s1}{StudyHours}\PY{l+s+s1}{\PYZsq{}}\PY{p}{,} \PY{l+s+s1}{\PYZsq{}}\PY{l+s+s1}{Grade}\PY{l+s+s1}{\PYZsq{}}\PY{p}{]}\PY{o}{.}\PY{n}{mean}\PY{p}{(}\PY{p}{)}\PY{p}{)}
\end{Verbatim}
\end{tcolorbox}

    \begin{Verbatim}[commandchars=\\\{\}]
       StudyHours      Grade
Pass
False    8.783333  38.000000
True    14.250000  73.142857
    \end{Verbatim}

    \begin{Verbatim}[commandchars=\\\{\}]
/tmp/ipykernel\_5397/2502225861.py:1: FutureWarning: Indexing with multiple keys
(implicitly converted to a tuple of keys) will be deprecated, use a list
instead.
  print(df\_students.groupby(df\_students.Pass)['StudyHours', 'Grade'].mean())
    \end{Verbatim}

    DataFrames are amazingly versatile and make it easy to manipulate data.
Many DataFrame operations return a new copy of the DataFrame, so if you
want to modify a DataFrame but keep the existing variable, you need to
assign the result of the operation to the existing variable. For
example, the following code sorts the student data into descending order
by Grade and assigns the resulting sorted DataFrame to the original
\textbf{df\_students} variable.

    \begin{tcolorbox}[breakable, size=fbox, boxrule=1pt, pad at break*=1mm,colback=cellbackground, colframe=cellborder]
\prompt{In}{incolor}{59}{\boxspacing}
\begin{Verbatim}[commandchars=\\\{\}]
\PY{c+c1}{\PYZsh{} Create a DataFrame with the data sorted by Grade (descending)}
\PY{n}{df\PYZus{}students} \PY{o}{=} \PY{n}{df\PYZus{}students}\PY{o}{.}\PY{n}{sort\PYZus{}values}\PY{p}{(}\PY{l+s+s1}{\PYZsq{}}\PY{l+s+s1}{Grade}\PY{l+s+s1}{\PYZsq{}}\PY{p}{,} \PY{n}{ascending}\PY{o}{=}\PY{k+kc}{False}\PY{p}{)}

\PY{c+c1}{\PYZsh{} Show the DataFrame}
\PY{n}{df\PYZus{}students}
\end{Verbatim}
\end{tcolorbox}

            \begin{tcolorbox}[breakable, size=fbox, boxrule=.5pt, pad at break*=1mm, opacityfill=0]
\prompt{Out}{outcolor}{59}{\boxspacing}
\begin{Verbatim}[commandchars=\\\{\}]
         Name  StudyHours  Grade   Pass
3       Rosie       16.00   97.0   True
10  Francesca       15.50   82.0   True
9    Giovanni       14.50   74.0   True
14      Jenny       15.50   70.0   True
21      Aisha       12.00   64.0   True
20     Daniel       12.50   63.0   True
11      Rajab       13.75   62.0   True
6    Frederic       11.50   53.0  False
19       Skye       12.00   52.0  False
1       Joann       11.50   50.0  False
0         Dan       10.00   50.0  False
4       Ethan        9.25   49.0  False
18      Anila       10.00   48.0  False
2       Pedro        9.00   47.0  False
7      Jimmie        9.00   42.0  False
12    Naiyana        9.00   37.0  False
16     Helena        9.00   36.0  False
17      Ismat        6.00   35.0  False
15     Jakeem        8.00   27.0  False
8      Rhonda        8.50   26.0  False
13       Kian        8.00   15.0  False
5       Vicky        1.00    3.0  False
\end{Verbatim}
\end{tcolorbox}
        
    \hypertarget{summary}{%
\subsection{Summary}\label{summary}}

NumPy and DataFrames are the workhorses of data science in Python. They
provide us ways to load, explore, and analyze tabular data. As we will
learn in subsequent modules, even advanced analysis methods typically
rely on NumPy and Pandas for these important roles.

In our next workbook, we'll take a look at how create graphs and explore
your data in more interesting ways.


    % Add a bibliography block to the postdoc
    
    
    
\end{document}
